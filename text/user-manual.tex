\chapwithtoc{Příloha A -- Uživatelská dokumentace}
Pro spuštění reimplementace Dungeon Masteru je zapotřebí alespoň:
\begin{itemize}
\item Windows x86 
\item DirectX 9.0c runtime  
\item .NET 4.6
\end{itemize}

Tento projekt si neklade za cíl udělat zcela kompletní a dobře hratelnou reimplementaci hry Dungeon Master.
Naopak se soustředí na dobrý návrh enginu jako takového. Z toho důvodu není herní zážitek nikterak
oslňující, nicméně jako demonstrace funkčnosti enginu poslouží dobře.
Hru je možné spustit souborem \ccc{/DungeonMaster/DungeonMasterEngine.exe}.

\section{Mechaniky ve hře}
Hráč reprezentuje vůdce skupiny šampionů. Může ovládat jejich pohyb,
předávat jim předměty, donutit je k boji, kouzlení, konzumaci jedlých předmětů nebo lektvarů.
Každý šampion má sadu vlastností a dovedností, v kterých je schopný se zdokonalovat získáváním zkušeností.
Zkušenosti lze získat bojem na prázdno, bojem proti nepřátelům, kouzlením nebo jen použitím některých přepínačů.
V herních úrovních jsou přepínače na zdech, které lze aktivovat stiskem klávesy \ccc{ENTER}. Dále jsou zde přepínače na 
podlaze, které lze aktivovat vstoupením na podlahu nebo hozením předmětu na daný spínač. Předměty lze
pokládat na zem, do výklenků a nebo je uložit na tělo šampiona či do jeho batohu. Přepínače pak mohou 
aktivovat nebo deaktivovat určité objekty ve hře, jako jsou například dveře, teleporty, jámy, otevírací
zdi, atd. Některé dveře je také možné rozbít útokem. Některé teleporty jsou pouze na konkrétní typy objektů.
Do jam lze spadnout, ale lze se z nich většinou teleportem dostat ven. V chodbách je tma, a proto je nutné 
používat pochodně -- což lze udělat vložením do jedné z ruk šampiona -- nebo je nutné vyvolat magické pochodně.

\section{Cíl hry}
Hra není úplně kompletní, proto ji není možné zcela dohrát. Z toho důvodu by se za cíl 
hry dalo považovat dostaní se do poslední herní úrovně, kde nebudou neimplementované funkce
bránit dalšímu postupu.

\section{Tutorial}

\imgxx{GPhalk}{obraz s šampionem}{GPapple}{předmět jablko}

Po spuštění hry se objevíme v první úrovní hry, kde si je nutné vybrat sadu šampionů. Mezi dlaždicemi se můžeme pohybovat
pomocí kláves W, S, A, D tj. dopředu, dozadu, doleva, doprava. Dále je možné se rozhlížet pomocí šipek na klávesnici.
Teď když víme, jak se pohybovat jděme chodbou dokud nenarazíme na nápis ,,HALL OF CHAMPION". Poté jděme hned 
vpravo směrem na pozici (7, 9), dokud nenarazíme na obraz na obrázku \ref{GPhalk}. Obrazy v sobě mají uložené šampiony, 
stisknutím klávesy \ccc{ENTER} při namířeném kurzoru na obraz, se zobrazí konzole, která nám vypíše statistiky šampiona 
a nabídne, zda ho chceme buď reinkarnovat nebo osvobodit -- v prvním 
případě lze specifikovat navíc jméno šampiona. V obou případech ale dojde k přidání šampiona do skupiny. 
Po přidání šampiona do skupiny se vpravo zobrazí jeho základní statistiky, tj. zdraví, výdrž a mana.
Konzoli lze zobrazit či skrýt stiskem klávesy \ccc{TAB}. Pojděme dále 
na pozici (9, 13), kde je další šampion, kterého přidejme do skupiny. Hráč může mít až 
čtyři šampiony, nicméně dva nám budou pro ukázku postačovat. První šampion je dobrý bojovník
a druhý dobrý čaroděj, takže si s nimi budeme moci ukázat jak souboj tak kouzla.


Vraťme se zpět k původnímu obrazu, vlevo od něj je brána a před ní na podlaze přepínač. Tento přepínač se aktivuje, pokud
na něj vkročí hráč s alespoň jedním šampionem. Projděme tedy bránou, dokud na zemi neuvidíme jablko \cc{GPapple}. 
V herních úrovních jsou všude roztroušeny potraviny, které jsou nutné k přežití skupiny. Ukázáním kurzorem na
jablko a stisknutím klávesy \ccc{ENTER} se jablko dostane do hráčovy ruky. S předměty v hráčově ruce lze provádět akce.
Zobrazme konzoli a napišme do ní příkaz \ccc{hand}, tím se zobrazí popis věci v ruce. Přidáním parametru \ccc{use}
lze vybrat šampiona, který jablko sní a sníží se tak jeho hlad. Jděme dále, dokud vpravo neuvidíme schody vedoucí
o úroveň níže. Na zemi lze nalézt lahev na vodu a svitek. Šampioni potřebují kromě jídla i pití -- do lahve lze nalít voda
z fontánek, které jsou různě rozmístěné po herních úrovních. Vezměme si tedy tuto lahev do ruky a pomocí následujícího
příkazu ji vložme do batohu druhého šampiona.

\begin{code}
> hand take
Select index:
0 HALK
1 TIGGY
> 1
Select index:
0 HeadStorageType
1 NeckStorageType
2 TorsoStorageType
3 ActionHandStorageType
4 ReadyHandStorageType
5 LegsStorageType
6 FeetsStorageType
7 BackPackStorageType
8 PouchStorageType
9 SmallQuiverStorageType
10 BigQuiverStorageType
> 7
\end{code}

Text svitku lze přečíst příkazem \ccc{hand}, pokud ho máme 
v hráčově ruce. Tento svitek říká, že kouzelnou pochodeň můžeme vyvolat pomocí symbolu ful.
Sekvence symbolů pro všechna kouzla mohou být nalezena v dokumentaci \cite{DMSpells}.

\begin{code}
> hand
Scrool "INVOKE FUL FOR A MAGIC TORCH"
\end{code}

Pro vyzkoušení kouzel
pojděme po schodech dolů, kde je naprostá tma. Kouzlit lze pomocí příkazu \ccc{spell} a to buď v interaktivním módu,
nebo pouze pomocí parametrů. Následující příklad ukazuje interaktivní mód, kdy je nejprve proveden dotaz
na šampiona, který bude kouzlo vyvolávat a dále už lze odříkávat jednotlivé symboly zakončené příkazem
\ccc{cast}. První symbol je tzv. power symbol, který určuje sílu kouzla. Každé vyvolání symbolu pak šampiona stojí manu. 
Každé kouzlo také specifikuje úrovně dovedností, které jsou nutné, aby bylo vyvolání úspěšné. Pokud má šampion
nižší úroveň, tak stále může kouzlo vyvolat ale s mnohem menší pravděpodobností. Úspěšný i neúspěšný pokus
o vyvolání kouzla šampionovi rovněž přidá zkušenosti -- avšak neúspěšný pokus daleko méně.

\begin{code}
> spell 
Select index:
0 HALK
1 TIGGY
> 1
Specify one of the following symbols: lo, um, on, ee, 
pal, mon (sorted by difficulty ascending).
Write symbol or cast it by "cast"
> lo
ManaProperty: 27 of 35 ; -8
Write symbol or cast it by "cast"
> ful
ManaProperty: 22 of 35 ; -5
Write symbol or cast it by "cast"
> cast
MagicalLightProperty: 409 of 2147483647 ; 409
Spell successfully casted!
\end{code}

Toto kouzle je slabé, proto můžeme jeho efekt navýšit opětovným vyvoláním.
Nyní tak ale proveďme druhou variantou pomocí parametrů, kde 
první specifikuje index šampiona, druhý power symbol a ostatní už jsou běžné symboly.

\begin{code}
>spell 1 lo ful
ManaProperty: 14 of 35 ; -8
ManaProperty: 9 of 35 ; -5
MagicalLightProperty: 818 of 2147483647 ; 409
Spell successfully casted!
\end{code}

Jak vidíme ve výstupu, na další vyvolání kouzla už nám nebude stačit mana. Rychlé obnovení zdraví, výdrže
a many lze provádět pomocí odpočinku. K němu lze využít příkaz \ccc{champion} s parametrem \ccc{sleep} --
probuzení lze provést příkazem \ccc{wake}, viz následující ukázka.

\begin{code}
> champion sleep
Write 'wake' to wake the group.
FoodProperty: 1541 of 2048 ; -2
WaterProperty: 1598 of 2048 ; -1
StaminaProperty: 750 of 750 ; 0
StaminaProperty: 443 of 450 ; -7
ManaProperty: 3 of 35 ; 1
...
FoodProperty: 1497 of 2048 ; -2
WaterProperty: 1483 of 2048 ; -1
StaminaProperty: 450 of 450 ; 0
FoodProperty: 1465 of 2048 ; -2
WaterProperty: 1560 of 2048 ; -1
StaminaProperty: 750 of 750 ; 0
wake
Party had just woken up.
\end{code}

Odpočinek zvyšuje u šampionů rychleji hlad a žízeň, proto je třeba doplňovat pití a jídlo.


\imgxx{GPtorch}{držák na pochodeň}{GPaltar}{VI Altar Rebirth}

Světlo je stále slabé, pojďme tedy zpět po schodech na pozici (4, 14), kde je na zdi pochodeň \cc{GPtorch}. Stiskem klávesy 
\ccc{ENTER} a kurzorem namířeným na držák pochodně se pochodeň přemístí hráči do ruky. Tím, že se pochodeň vloží 
do jedné z ruk některého šampiona, se rozsvítí. Intenzita světla, které pochodeň produkuje postupně klesá, až úplně zhasne.
Vložme tedy pochodeň například prvnímu šampionovi do druhé ruky následujícím
příkazem.

\begin{code}
> hand take
Select index:
0 HALK
1 TIGGY
> 0
Select index:
0 HeadStorageType
1 NeckStorageType
2 TorsoStorageType
3 ActionHandStorageType
4 ReadyHandStorageType
5 LegsStorageType
6 FeetsStorageType
7 BackPackStorageType
8 PouchStorageType
9 SmallQuiverStorageType
10 BigQuiverStorageType
> 4
\end{code}

Ještě než sejdeme opět dolů, podívejme se na výklenek \cc{GPaltar} na pozici (4, 17). Pokud nějaký šampion zemře, zbudou po něm
na zemi kosti. Do takto vypadajících výklenků lze pak tyto kosti vložit,  čímž dojde k oživení šampiona.
Dále v chodbě jsou pak dveře, které lze otevřít červeným tlačítkem na jejich okraji.


\imgxx{GPmummy}{mumie útočící na hráče}{GPkey}{dveře na klíč}

Pojďme ale zpět o úroveň níž a pak vpravo až ke dveřím na pozici (6, 14), za kterými se skrývá mumie \cc{GPmummy}. 
Po otevření dveří tlačítkem na nás zaútočí. Když bude mumie na dlaždici před hráčem lze na ní aplikovat útok na blízko.
Útok se provede příkazem \ccc{fight}, který je rovněž k dispozici v interaktivní či parametrické formě. Nejprve
je třeba vybrat šampiona, s kterým chceme útok provést a následně akci, kterou lze provést se zbraní šampiona.
Používá se vždy zbraň v akční ruce. Po provedení akce se zbraní se šampionovi sníží výdrž, pokud je výdrž na příliš
malé úrovni, mají pak útoky menší poškození. Po provedení akce zároveň šampion nemůže nějakou dobu útočit.

\begin{code}
> fight 
Select index:
0 HALK
1 TIGGY
> 0
Select index:
0 Throw
1 Bash
> 1
HealthProperty: 0 of 20 ; -20
StaminaProperty: 746 of 750 ; -4
StaminaProperty: 736 of 750 ; -10

Action available.
\end{code}

V tomto případě jsme měli štěstí, jelikož šampion zabil mumii na jeden útok.
Pokud by bylo útoků potřeba provést více, existuje také zmíněná parametrická varianta příkazu, viz následující
ukázka útoku na jinou nepřátelskou entitu. První parametr opět specifikuje index šampiona, další pak index akce.

\begin{code}
fight 0 1
HealthProperty: 37 of 56 ; -11
StaminaProperty: 731 of 750 ; -6
StaminaProperty: 729 of 750 ; -2

Action available.
\end{code}

Nyní projděme dveře na pozici (1, 14), až dojdeme ke dveřím \cc{GPkey}, které nemají tlačítko. 
Vedle nich lze na zemi nalézt klíč, který lze vložit do zdířky vedle na stěně. Tím
jsou dveře otevřeny. Na pozici (3,  22) je potom fontánka \cc{GPfountain}, v které si lze doplnit vodu do lahve nebo prázdné baňky.
 Další klíč je na pozici (1, 23) a lze s ním otevřít dveře na pozici
(4, 19), za kterými jsou dveře další. Pro jejich otevření musíme získat klíč na pozici (7, 23).
Cestu k tomuto místu blokují dveře, které lze otevřít pákou na pozici (4, 25) a jáma \cc{GPpit}, kterou lze otevřít
pákou na pozici (6, 21). 

\imgxx{GPfountain}{fontánka s vodou}{GPpit}{jáma zavíratelná pákou}

Tímto jsme prošli základní mechaniky hry, další postup už je na hráči.

\section{Seznam příkazů a jejich použití}
\begingroup
\fontfamily{lmtt}\selectfont
\begin{itemize}
\item hand - použití: hand [put\textpipe putsub\textpipe take\textpipe takesub]
	\begin{itemize}
	\item bez parametrů:  zobrazí textový popis předmětu v ruce
	\item put: vloží předmět z šampionova inventáře do hráčovi ruky 
	\item putsub: vloží předmět z truhly v šampionově inventáři do hráčovy ruky
	\item take: vezme předmět z hráčovy ruky a vloží jej to šampionova inventáře 
	\item takesub: vezme předmět z hráčovy ruky a vloží jej do truhly v šampionově inventáři 
	\end{itemize}

\item champion - použití: champion list\textpipe sleep
	\begin{itemize}
	\item list: zobrazí seznam šampionů a pro vybraného šampiona zobrazí jeho dovednosti 
	\item sleep: přivede skupinu šampionů k odpočinku -- probuzení se provede příkazem wake
	\end{itemize}

\item spell - použití: spell [championIndex symbolSequence]
	\begin{itemize}
	\item bez parametrů: interaktivní výběr šampiona a vyvolání kouzla
	\item s parametry: šampion daný indexem championIndex vyvolá kouzlo  definované sekvencí symbolů symoblSequence
	\end{itemize}

\item fight - použití: fight [champoinIndex] [actionIndex] 
	\begin{itemize}
	\item bez parametru: interaktivní výběr šampiona a akce
	\item s parametry: šampion s indexem championIndex provede akci s indexem actionIndex
	\end{itemize}

\item help - použití: help [commandToken]
	\begin{itemize}
	\item bez parametrů: zobrazí všechny příkazy 
	\item commandToken: zobrazí pomocnou zprávu k příkazu commandToken 
	\end{itemize}
\end{itemize}

\endgroup

\chapwithtoc{Příloha B -- Struktura přiloženého CD}
Zdrojové kódy jsou určeny pro Visual Studio 2015 s doinstalovaným MonoGame SDK 3.4.
\begin{itemize}
\item \ccc{/SourceCodes} -- obsahuje solution s oběma následujícími projekty.
	\begin{itemize}
	\item \ccc{/DungeonMaster} -- projekt s parserem vstupních dat z originální hry Dungeon Master.
	\item \ccc{/DungeonMasterEngine} -- projekt s reimplementovaným enginem hry Dungeon Master.
	\end{itemize}
\item \ccc{/Framework} -- obsahuje použitý instalátor MonoGame SDK 3.4.
\item \ccc{/DungeonMaster} -- obsahuje všechny soubory reimplementace hry Dungeon Master.
	\begin{itemize}
	\item \ccc{/DungeonMasterEngine.exe}  -- binárka spouštějící hru.
	\end{itemize}
\item \ccc{/prace.pdf} -- text této práce.
\item \ccc{/README.txt} -- informace o struktuře CD.
\end{itemize}