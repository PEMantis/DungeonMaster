\chapter{Uživatelská dokumentace}
Pro spuštění reimplementace Dungeon Masteru je zapotřebí:
\begin{itemize}
\item Windows Vista a novější 
\item DirectX 9.0c runtime  
\item .NET 4.6
\end{itemize}

Tento projekt si neklade za cíl udělat zcela kompletní a dobře hratelnou reimplementaci Dungeon Master.
Naopak se soustředí na dobrý návrh enginu jako takového. Z toho důvodu není herní zážitek nikterak
oslňující, nicméně jako demonstrace funkčnosti enginu poslouží dobře.
Hru je možné spustit souborem \ccc{/DungeonMaster/DungeonMasterEngine.exe}.

\section{Mechaniky ve hře}
Hráč reprezentuje vůdce skupiny bojovníků známých jako šampióni. Může ovládat jejich pohyb, 
předávat jim předměty, donutit je k boji, ke kouzlení, ke konzumaci jedlých předmětů nebo lektvarů.
Každý bojovník má sadu vlastností a dovedností, v kterých je schopný se zdokonalovat získáváním zkušeností.
Zkušenosti lze získat bojem na prázdno, bojem proti nepřátelům, kouzlením, nebo jen použitím některých přepínačů.
V dungeonu jsou přepínače na zdech, které lze aktivovat kliknutím myši. Dále jsou zde přepínače na 
podlaze, které lze aktivovat vstoupením na podlahu nebo hozením předmětu na daný spínač. Předměty lze
pokládat na zem, do výklenků a nebo je uložit na tělo bojovníka či do jeho batohu. Přepínače pak mohou 
aktivovat nebo deaktivovat určité objekty ve hře jako jsou například dveře, teleporty, jámy, otevírací
zdi, atd. Některé dveře je také možné rozbít útkem. Některé teleporty jsou pouze na konkrétní typy objektů.
Do jam lze spadnout, ale lze se z nich většinou teleportem dostat ven. V chodbách je tma a proto je nutné 
používat pochodně -- což lze vložením do jedné z ruk šampiona -- nebo je nutné vyvolat magické pochodně.

\imgx{dm-new-screen}{Ilustrace reimplementace hry Dungeon Master}

\section{Cíl hry}
Hra není úplně kompletní, proto ji není možné zcela dohrát. Z toho důvodu by se za cíl 
hry dalo považovat dostaní se do poslední herní úrovně, kde nebudou neimplementované funkce
bránit dalšímu postupu.

\section{Ovládání}
Pohybovat skupinou lze pomocí kláves W, S, A, D tj. dopředu, dozadu, doleva, doprava. Dále se je možné rozhlížet pomocí šipek na klávesnici.

\subsection{Aktivace přepínačů a sbírání předmětů}
Přepínače lze aktivovat namířením ukazatele myši na daný objekt a kliknutím nebo stiskem klávesy \ccc{ENTER}. Pokud
se pod kurzorem vyskytuje předmět nebo  přepínač, který může vložit nějaký předmět hráčovi do ruky, tak se tak provede, pokud je
hráčova ruka prázdná. Pokud hráč již v ruce něco má, provede se opačná akce tj. předmět se položí na dané místo nebo je 
pohlcen přepínačem. Ostatní interakce probíhá přes konzoli. Konzole se aktivuje stiskem klávesy \ccc{TAB} a deaktivuje opětovným
stisknutím tytéž klávesy. Příkaz \ccc{hand} zobrazí popis předmětu v hráčově ruce. Přidáním parametru \ccc{take} se provede
uložení předmětu do dotazovaného inventáře daného bojovníka. Přidáním parametru \ccc{put} se naopak vloží dotazovaný objekt z inventáře
do ruky. Dalším možným parametrem příkazu je \ccc{takesub} resp. \ccc{putsub}, pomocí kterého lze uložit předmět z ruky do
truhly v inventáři nějakého šampiona resp. vložit předmět z truhly šampiona do ruky. Posledním možný parametrem příkazu \ccc{hand} je
parametr \ccc{use}, který se pokusí aplikovat předmět v ruce na zvoleného šampiona. Takovým způsobem lze použít jídlo, lektvary, nebo voda.

\subsection{Souboj}
Pro souboj je nejprve nutné vložit zbraň to akční ruky. Což je možné provést příkazy z minulé sekce.
Samotný souboj probíhá potom skrze příkaz \ccc{fight}, kdy je interaktivně vybrán bojovník a způsob útoku.
Příkaz lze také použít s dvěma číselnými parametry, kde první identifikuje pořadí šampiona a druhý pořadí akce.
Kromě boje na blízko lze ještě vyvolávat útočná kouzla. K tomu slouží příkaz \ccc{spell}, kterému se jako první parametr předá index šampiona a pak buď
jako parametry jména symbolů, nebo při nespecifikování parametrů příkazu lze šampiona a symboly zadat interaktivně.

\subsection{Odpočinek}
Posledním příkazem je \ccc{champion}, který může mít následující parametry:
\begin{itemize}
\item \ccc{list} -- vypíše jména šampionů.
\item \ccc{sleep} -- uspí šampiony a tím dojde krychlejší regeneraci vlastností.
\item \ccc{wake} -- probudí šampiony ze spánku. 
\end{itemize}

Pro více informací o příkazech je možné napsat příkaz \ccc{help} do konzole.